\subsection{EBNF - Arith}
This example is similar to the previous two, but we added a few more productions and lexical classes.
We do this by adding unary operations and single length character variables to the grammar definiton. The most important 
differences in this example are the productions that utilize the optional and production ors notion and the
lexical definition that uses the keyword \textit{char}.

The first difference is the optional notation. Below we can see that \textit{unary\_op} is optional because it is
between brackets and anything in between square brackets will be transformed into the OCaml optional type
by gt when generating the parser. \\
%
%
\begin{gt}
Fact : factor -> [ unary_op ] element.
\end{gt}\ \\
%
%
The next major change is the use of vertical bars to denote a type has multiple definitions. This 
can be very useful because it cuts back on defining extra productions. Instead of 
having three different productions for \textit{element} we create one and utilize gt's EBNF notation to define the 
several different possibilities for element.\\
%
%
\begin{gt} 
Element : element -> INT | VAR | LPAREN expr RPAREN.
\end{gt}\ \\
%
%
The last major change is the use of the keyword \textit{char}. If the regular expression for a lexical
definition has a maximum length of one then this keyword is needed. \\
%
%
\begin{gt} 
VAR = char {{ ['a'-'z''A'-'Z']}}.
\end{gt}\ \\
%
%
Gt also supports basic definitions for Emacs and Gedit syntax modes. It is very simple to define basic syntax
modes for the user created language using gt. The first thing the user needs to do is to define a file extensions.
After a file extension is defined the user can map terminals that are not in the abstract syntax tree to
predefined keywords. If the user does not define a syntax mode every terminal that is not in the abstract syntax tree
will be mapped to \textit{keyword} and the file extension will be the \textit{grammar\_name}. \\
\begin{gt}
file_ext = "gra"
keywords = 
	("x","+","-") : special
	(")","(") : keyword
\end{gt} 
%
Below is the whole grammar definition for the \textit{arith} grammar. The file name is \textit{ebnf-example.gra} and it can 
be located in the \textit{gt/tests/ebnf} directory.\\
%
%
\begin{gt}
arith

Term : expr -> factor { op factor }(right,>=0).
Fact : factor -> [ unary_op ] element.
Element : element -> INT | VAR | LPAREN expr RPAREN.

Op : op -> PLUS | TIMES.
UnaryOp : unary_op -> MINUS.

PLUS = "+".
TIMES = "x".
MINUS = "-".
LAPREN = "(" .
RPAREN = ")".
INT = {{ ['0'-'9']+ }}.
VAR = char {{ ['a'-'z''A'-'Z']}}.

file_ext = "gra"
keywords = 
	("x","+","-") : special
	(")","(") : special
\end{gt} 


\subsubsection{Generated Functions}
Below is the generated parse file for the grammar defined above.\\
\lstinputlisting[language=ML]{./examples/ebnf/simple/expr_parse.mly}\ \\
%
Below is the generated lex file for the grammar defined above. More general 
descriptions of these files can be found in Part II of this manual. \\
\lstinputlisting[language=ML]{./examples/ebnf/simple/expr_lex.mll}

\subsubsection{Syntax Trees}

Now that the parser has been generated by gt it is ready to be used. A useful feature
of gt is being able to generate visual syntax trees. The example will show the 
syntax tree that is generated by the above grammar. Figure 2.1 is an example of a 
that was generated by gt's gviz function. \\ \\

\begin{figure}[h!]
  \centering
  \includegraphics[width=3in]{./examples/bnf/simple/simple.png}
  \caption{Syntax Tree for\\ \textit{x + x + x + x}}
\end{figure}




