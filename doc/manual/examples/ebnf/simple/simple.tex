\subsection{EBNF - Simple}

The following example will go over the grammar definition in the file \textit{ebnf-simple-exprs.gra} which can be found in 
\textit{gt/tests/}. The goal of this grammar is to create parser that will successfully parse simple
multiplication and addition expressions. An example of a parseable string would be:
\\\\
\begin{minipage}[t]{.8\linewidth}
0 + 09 * 23 + (48 * 33 + (88 + 23)) + 676 
\end{minipage}
\\\\
The name of this grammar is \textit{expr}. The start symbol is also \textit{expr} and is described as an 
\textit{term} with zero or many terms separated by a \textit{PLUS} after the initial \textit{term}. 
Where the lexical class \textit{PLUS} is '+' and a \textit{term} is zero or many \textit{factor TIMES} with a 
mandatory \textit{factor}. The lexical class \textit{TIMES} is defined as '*' and a \textit{factor} can either 
be an \textit{expr} surrounded by parenthesis or an \textit{INT}. Below is a walkthrough of this grammar.

Again the grammar name has been defined as \textit{expr} and in this example a \textit{line\_comment}
has not been defined and will be set to the default comment delimeter, \textit{\#}. Next, we need to define 
an \textit{expr}. Below is a gt definition for the grammar explained earlier. \\
%
%
%
\begin{gt} 
expr

Plus : expr -> term PLUS term.
\end{gt}\ \\
Since multiplication is left associative this example uses the built in feature of gt to denote this.
The following production name has been defined as \textit{Mult}.\\
\begin{gt} 
(* The following can also be written as:
Mult: term -> { factor TIMES }* factor *)
Mult : term -> { factor TIMES }(left,>=0) factor.
\end{gt}\ \\
%
%
The last production we need to define is for factor. As explained earlier a factor can either be 
an can be an expr with parenthesis surrounding the expr or an INT.\\
\begin{gt}
Expr : factor -> LPAREN expr RPAREN.
Int : factor -> INT.
\end{gt}\ \\
%
%
%
Next, we need to define the lexical classes for PLUS, TIMES, RPAREN, LPAREN and INT. PLUS and 
TIMES, RPAREN, LPAREN are lexical classes that are not in the abstract syntax tree. For 
INT we have to create a regular expression in OCaml syntax. Putting 
these steps together we get the final grammar definition which is can be found in the ebnf-simple-exprs.gra file.\\
%
\begin{gt} 
expr

Plus : expr -> term PLUS term.

Mult : term -> { factor TIMES }(left,>=0) factor.

Expr : factor -> LPAREN expr RPAREN.
Int : factor -> INT.

PLUS = "+".
TIMES = "x".
LAPREN = "(" .
RPAREN = ")".
INT = {{ ['0'-'9']+ }}.
\end{gt}
%
%
\subsubsection{Generated Functions}
Below is the generated parse file for the grammar defined above.\\
\lstinputlisting[language=ML]{./examples/ebnf/simple/expr_parse.mly}\ \\
%
Below is the generated lex file for the grammar defined above. More general 
descriptions of these files can be found in Part II of this manual. \\
\lstinputlisting[language=ML]{./examples/ebnf/simple/expr_lex.mll}

\subsubsection{Syntax Trees}

Now that the parser has been generated by gt it is ready to be used. A useful feature
of gt is being able to generate visual syntax trees. The example will show the 
syntax tree that is generated by the above grammar. Figure 2.1 is an example of a 
that was generated by gt's gviz function. \\ \\

\begin{figure}[h!]
  \centering
  \includegraphics[width=3in]{./examples/bnf/simple/simple.png}
  \caption{Syntax Tree for\\ \textit{x + x + x + x}}
\end{figure}








