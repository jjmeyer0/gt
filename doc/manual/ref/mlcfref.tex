\section{Reference: Generating Functions with MLCF}

Below is a general description of mlcf's (meta language for common functions) grammar. The actual grammar definition can be
found in the mlcf\_parse.mly file in the \textit{src} directory of mlcf or in the mlcf.gra file in the \textit{mlcf} directory. 
It is not required to understand mlcf. A user will never have to use this tool unless one of the files that is created 
by it was deleted or broken. However, if a user would like to extend gt, mlcf may be usefule. Mlcf is basically a template for iterating
through the productions of a gt grammar. It also, has common predefined functions that can be used when defining an mlcf function 
For examples of mlcf function please look at the mlcf in the \textit{mlcf/functions/} directory. The mlcf functions modify 
\textit{dump\_mlcf\_*} files in \textit{gt/src/}. The following grammar definition is based on mlcf.gra:\\
\[
\begin{array}{lll}
\textit{program} & \to^m & \ [\ \textit{ID}\ \textit{ENDEACH}\ =\ \textit{IQUOTED}\ ]^m\ [\ \textit{MLGLOBAL}\ ]^m\ \textit{const}\ \{\ \textit{body}\ \}*^m \\


\textit{const} & \to^m & \textit{ID}\ \textit{(}\ \textit{ID} \textit{,}  \textit{ID}\ \textit{...}\ \textit{,}  \textit{ID}\ \textit{)}\ \textit{-$>$} \\ 
\textit{body} & \to^m & \textit{OCAMLGLOBAL} \\
\textit{body} & \to^m & \textit{METACODE}\ \textit{ID} \ \textit{SEPBY} \\ \\

\textit{ENDEACH} & = & "\textnormal{end\_of\_each\_match}" \\

\textit{MLGLOBAL} & = & \textnormal{starts with \{\{, ends with \}\} and allows anything in between } \\
\textit{METACODE} & = & \textnormal{starts with (\~{}, ends with \~{}$|$ and allows anything in between } \\
\textit{SEPBY} & = & \textnormal{starts with \~{}$|$, ends with \~{}) and allows anything in between } \\
\textit{ID} & = & \textnormal{A string that starts with an upper/lower-case letter followed } \\
\ & \ & \textnormal{by numbers, letters and/or \_} \\
\textit{IQUOTED} & = & \textnormal{standard string literal} 
\end{array}
\]

\subsection{Example: Pretty Printer function}

There are four basic parts to an mlcf definition: the function name, a code block before iterating through the productions,
the body of the iteration function and a code block that is to be executed after iterating through the productions. The function
name -- below it is defined as \textit{pp} -- is used in the filename of the dumped function and in naming some predefined
functions that are used when defining mlcf functions. 

The code blocks can have any valid OCaml code in them. The iteration block --
where most of the OCaml code is written -- has three parts: the code block, the counter variable and with what to separate
each iteration. There are a few predefined functions is\_list, is\_opt, is\_terminal, is\_in\_ast and string\_of\_terminal that can 
be used in any mlcf program definition. Below the function pp\_i refers to the pretty printer function name that will be used for
the current element. This function will always be named \textit{function\_name}\_i. The function x\_i refers to the element to 
pretty print and it will be named with the first letter of \textit{x1} followed by \textit{\_i}. \\

\noindent Below is the complete mlcf definition for the pretty printer function. 
\begin{lstlisting} [language=ML]
pp

PrettyPrinter (P, x1,...,xn)-> 
{{ emit_pattern 0; os " -> "; }}

(~
if(is_terminal) then (
	os ("print_new_line os to_pretty_print (fst d); ");
	if(is_in_ast) then (
		os " pp_terminal os to_pretty_print "
	) else (
		os ("print_new_line os to_pretty_print (fst "^x_i^"); ");
		os "os "; os string_of_terminal
	);
) else (
	os pp_i; os " os to_pretty_print "
);

if(is_cur_eq_first_nt && (is_list || is_opt)) then (
	os ( " (d,"^x_i^")" );
) else if not(is_terminal && not is_in_ast) then (
	os ( x_i );
);

if(is_terminal) then os "; os \" \" ";

~| i ~| ; ~)

{{ os " () "; }}
\end{lstlisting}