\section{Tutorial}
In this section we will explain the basic format of a gt grammar definition. Later we will show complete
examples. To look at specific grammar definitions please read Chapter 2, Examples. Alternatively,
there are more examples in the \textit{gt/tests} directory.

\subsection{GT File Format}
A grammar file has three basic parts: the name of the grammar, a line comment declaration and the body. 
The body of a grammar file is broken up into two sections: the productions and the lexical definitions.

The first part of a grammar definition is the grammar name. The grammar name will be used to prefix the names
of the files generated by gt. It will also be the name of the default compiled executable, which parses string 
to syntax trees and can then print them back. The next part of the definition is where a user can define a line-comment. 
This is optional and if left out the default line-comment delimiter is the hash symbol, \textit{\#}. Lastly, the body 
of a grammar definition is where most of the work in defining a context-free grammar is done. The body is split into 
two sections, one for production definitions and the other for lexical class definitions.\\
\begin{gt}
Expr : factor -> RPAREN expr LPAREN.
Int : factor -> INT.
\end{gt}\ \\
In the example above the first ID, \textit{Expr}, is used by gt when defining OCaml constructors and therefore needs
to be unique for each production and start with an uppercase letter. The next ID is used by gt for the name of the corresponding
Ocaml types and should start with a lowercase letter. Next are the elements of the 
production. The elements of a production will define what a production can shift or reduce to. Elements
will be used by gt to define the corresponding OCaml types of the current production. The following is the OCaml
code that would be generated by gt for the gt definition shown above. These types are defined and used for syntax trees. \\
\begin{lstlisting} [language=ML]
(* pd holds position data and file name. *) 
type __terminal__ = (pd * string);;
type __term_not_in_ast__ = pd;;

type factor =  
    | Int of pd * __terminal__
    | Expr of pd * __term_not_in_ast__ * expr * __term_not_in_ast__;;
\end{lstlisting} 

In the OCaml code above \textit{\_\_terminal\_\_} represents a lexical definition that is in the abstract
syntax tree such as \textit{\{\{['0'-'9']+\}\}}. Where as lexical definitions such as, \textit{"+"} are represented by the 
\_\_terminal\_not\_in\_ast\_\_ type. The next section of the body is where the lexical classes are defined. These can
be defined as either a string (eg. "+") or as a regular expression (e.g. \{\{ ['0'-'9']+ \}\}).
The regular expression should be defined using OCaml regular expression syntax.
Gt will store the lexical definitions that are in the abstract syntax tree with the string and position data of the corresponding 
lexical definition. Lexical definitions that are not in the abstract syntax tree only store the position data. Below is a simple 
example of a grammar definitions's lexical classes.\\ 

\begin{gt} 
PLUS = "+".
INT = {{ ['0'-'9']+ }} .
\end{gt} \ \\
\noindent The previous lexical definitions are compiled to the following OCaml 
lexical classes. These can be found in \textit{grammar\_name}\_lex.mll.\\

\begin{gt} 
...
| [ '0' - '9' ]+ as str { INT((StatSym_util.cur_pd(),str)) }
| "+" PLUS(StartSym_util.cur_pd())
...
\end{gt} 

