\section{Installing Grammar Tool (gt)}

In this section there are a few basic steps to follow in order to install
and use gt from the terminal. This guide is assuming that 
the user has already obtained gt. If you do not have it there is a list
of links in Part II: Reference Manual where gt may be downloaded. First, make sure OCaml 
is installed on your computer. Gt is known to work with version 3.12.0 of OCaml, but it should work
with older versions as it only uses the standard libraries. To test if your machine has 
OCaml open a terminal and type the following command.\\

\begin{lstlisting}[language=bash]
ocaml -version
\end{lstlisting}\ \\
\noindent If OCaml your current machine has OCaml installed you should get a message saying
the version of OCaml is found. If you do not have it go to the following page, download the latest 
version and follow the directions that are included in order to install OCaml. \\

\begin{minipage}[t]{.8\linewidth}
http://caml.inria.fr/download.en.html
\end{minipage}\\

\noindent The following commands do \emph{not} need to be executed unless files have been 
deleted or altered. First, navigate to \textit{gt/} from the terminal. Next, execute the following 
commands: \\

\begin{lstlisting}[language=bash]
make clean 
make update 
\end{lstlisting}\ \\
\noindent Lastly, if the user plans to view syntax trees then Graph-Viz
is needed. To obtain Graph-Viz follow the link below to download
the version that works on your machine. The user will still be able
to generate the syntax trees without Graph-Viz, but they will not
be able to view or export them without downloading and installing Graph-Viz. \\

\begin{minipage}[t]{.8\linewidth}
http://graphviz.org/Download.php
\end{minipage}\\

\noindent A basic use of gt is shown as an example. There are other flags that 
are recognized by gt. To learn about these use the \textit{./gt -help} command. 
Below \textit{file} is referring to a path of a parseable grammar file. To generate
a parser from the terminal, navigate to \textit{gt/src/}. Then execute the following 
commands. \\

\begin{lstlisting}[language=bash]
make clean 
make 
# for usage type ./gt -usage
./gt file 
make -f grammar_name_Makefile 
\end{lstlisting}\ \\
\noindent The command \textit{make clean} will remove all of the compiled OCaml
files. The \textit{make} command will compile gt and generate an executable 
called gt, the grammar tool that will generate a parser from a grammar definition. To 
use the newly compiled executable use the command \textit{./gt file} where \textit{file} is a path to a parsable grammar 
definition. Now that gt has generated all of the necessary files, use the command \textit{make emitted} or 
\textit{make -f grammar\_name\_Makefile} to compile your parser. Once this has been done an executable with the same name as
the \textit{grammar\_name} in the corresponding grammar definition has been created. 

Now that executable is ready to be used, below is a walkthrough on how to use the
newly generated parser. The following is how to use the compiled executable. This can be seen by typing 
\textit{./grammar\_name -help} in the terminal.\\
\begin{lstlisting}[language=bash]
./grammar_name [options] <file> 
    The options are:
       -p      Reprints the contents of <file> to
               <file>pp.txt. Without this option
               the contents are printed to the terminal.
       -a      Prints a text view of <file>'s AST to 
               <file>ast.txt.
       -g      Outputs a Graphviz file named <file>gviz.dot
               that contains a visual representation of <file>'s
               syntax tree.
       -help   prints this
\end{lstlisting}







